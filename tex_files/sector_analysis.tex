% This LaTeX was auto-generated from MATLAB code.
% To make changes, update the MATLAB code and export to LaTeX again.

\documentclass{article}

\usepackage[utf8]{inputenc}
\usepackage[T1]{fontenc}
\usepackage{lmodern}
\usepackage{graphicx}
\usepackage{color}
\usepackage{hyperref}
\usepackage{amsmath}
\usepackage{amsfonts}
\usepackage{epstopdf}
\usepackage[table]{xcolor}
\usepackage{matlab}

\sloppy
\epstopdfsetup{outdir=./}
\graphicspath{ {./sector_analysis_images/} }

\begin{document}

\matlabtitle{Analysis of Antarctic sectors}

\begin{par}
\begin{flushleft}
There are three sectors available:
\end{flushleft}
\end{par}

\begin{enumerate}
\setlength{\itemsep}{-1ex}
   \item{\begin{flushleft} SA := South Africa,  \end{flushleft}}
   \item{\begin{flushleft} WS := Weddell Sea, \end{flushleft}}
   \item{\begin{flushleft}  and EA := East Antarctica \end{flushleft}}
\end{enumerate}

\matlabheading{Table of contents}

\begin{enumerate}
\setlength{\itemsep}{-1ex}
   \item{\begin{flushleft} Preamble \end{flushleft}}
   \item{\begin{flushleft} Ice concentration (aice) \end{flushleft}}
   \item{\begin{flushleft} Pancake ice concentration \end{flushleft}}
   \item{\begin{flushleft} Pancake ice width \end{flushleft}}
   \item{\begin{flushleft} Pancake ice growth  \end{flushleft}}
   \item{\begin{flushleft} SWH presence \end{flushleft}}
   \item{\begin{flushleft} SWH distance into the ice \end{flushleft}}
   \item{\begin{flushleft} Ice velocities \end{flushleft}}
   \item{\begin{flushleft} Wind and ice relationship \end{flushleft}}
\end{enumerate}

\matlabheading{1. Preamble}

\begin{matlabcode}
%% Read in the data.
clear
close all
addpath functions
%ncdisp(filename)

% Parameters
sector = "SA";
grid = 'gx1';
filedir = 'cases/12monthswim/history/iceh.2005-09-30.nc';
[lat,lon,row] = grid_read(grid);

% Make sector mask
[len,wid] = size(lat);
\end{matlabcode}

\matlabheading{2. Ice area}

\begin{matlabcode}
variable = "aice";
sector_data = data_format_sector(filedir,variable,sector);
%data = data(sector_mask);
%color_map = seaicecolormap();
latitude = [-90,-60];
longitude = [10,50];
figure(1)
w = worldmap('world');
    axesm miller; %, eqaazim eqdazim vperspec, eqdazim flips the x-axis, and y-axis to eqaazim. cassini
    setm(w, 'Origin', [0 0 0]);
    setm(w, 'maplatlimit', [-70,-40]);
    setm(w, 'maplonlimit', [10,50]);
    setm(w, 'meridianlabel', 'on')
    setm(w, 'parallellabel', 'on')
    setm(w, 'mlabellocation', 10);
    setm(w, 'plabellocation', 10);
    setm(w, 'mlabelparallel', 0);
    setm(w, 'grid', 'on');
    setm(w, 'frame', 'on');
    setm(w, 'labelrotation', 'on')
    pcolorm(lat,lon,sector_data)
    land = shaperead('landareas', 'UseGeoCoords', true);
    geoshow(w, land, 'FaceColor', [0.5 0.7 0.5])
    colorbar
    caxis(colorlims(variable));
    title(strcat("AICE in the ", sector_name(sector)), 'interpreter','latex','FontSize', 14)
\end{matlabcode}
\begin{center}
\includegraphics[width=\maxwidth{56.196688409433015em}]{figure_0.png}
\end{center}

\matlabheading{3. Pancake ice concentration}

\begin{matlabcode}
variable = "fsdrad";
sector_data = data_format_sector(filedir,variable,sector);
%data = data(sector_mask);
%color_map = seaicecolormap();
%colormap turbo
latitude = [-90,-60];
longitude = [10,50];
figure(2)
w = worldmap('world');
    axesm miller; %, eqaazim eqdazim vperspec, eqdazim flips the x-axis, and y-axis to eqaazim. cassini
    setm(w, 'Origin', [0 0 0]);
    setm(w, 'maplatlimit', [-70,-40]);
    setm(w, 'maplonlimit', [10,50]);
    setm(w, 'meridianlabel', 'on')
    setm(w, 'parallellabel', 'on')
    setm(w, 'mlabellocation', 10);
    setm(w, 'plabellocation', 10);
    setm(w, 'mlabelparallel', 0);
    setm(w, 'grid', 'on');
    setm(w, 'frame', 'on');
    setm(w, 'labelrotation', 'on')
    pcolorm(lat,lon,sector_data)
    land = shaperead('landareas', 'UseGeoCoords', true);
    geoshow(w, land, 'FaceColor', [0.5 0.7 0.5])
    colorbar
    caxis(colorlims(variable));
    title(strcat("FSDRAD in the ", sector_name(sector)), 'interpreter','latex','FontSize', 14)
    %colormap turbo
\end{matlabcode}

\matlabheading{4. Pancake ice width}


\vspace{1em}
\matlabheading{5. Pancake ice growth}

\begin{matlabcode}
variable = "fsdrad";
sector_data = data_format_sector(filedir,variable,sector);
%data = data(sector_mask);
%color_map = seaicecolormap();
%colormap turbo
latitude = [-90,-60];
longitude = [10,50];
figure(5)
w = worldmap('world');
    axesm miller; %, eqaazim eqdazim vperspec, eqdazim flips the x-axis, and y-axis to eqaazim. cassini
    setm(w, 'Origin', [0 0 0]);
    setm(w, 'maplatlimit', [-70,-40]);
    setm(w, 'maplonlimit', [10,50]);
    setm(w, 'meridianlabel', 'on')
    setm(w, 'parallellabel', 'on')
    setm(w, 'mlabellocation', 10);
    setm(w, 'plabellocation', 10);
    setm(w, 'mlabelparallel', 0);
    setm(w, 'grid', 'on');
    setm(w, 'frame', 'on');
    setm(w, 'labelrotation', 'on')
    pcolorm(lat,lon,sector_data)
    land = shaperead('landareas', 'UseGeoCoords', true);
    geoshow(w, land, 'FaceColor', [0.5 0.7 0.5])
    colorbar
    caxis(colorlims(variable));
    title(strcat("FSDRAD in the ", sector_name(sector)), 'interpreter','latex','FontSize', 14)
\end{matlabcode}
\begin{center}
\includegraphics[width=\maxwidth{56.196688409433015em}]{figure_1.png}
\end{center}
\begin{matlabcode}
    %colormap turbo
\end{matlabcode}

\vspace{1em}

\matlabheading{6. SWH presence}

\begin{matlabcode}
variable = "wave_sig_ht";
sector_data = data_format_sector(filedir,variable,sector);
%data = data(sector_mask);
%color_map = seaicecolormap();
%colormap turbo
latitude = [-90,-60];
longitude = [10,50];
figure(6)
w = worldmap('world');
    axesm miller; %, eqaazim eqdazim vperspec, eqdazim flips the x-axis, and y-axis to eqaazim. cassini
    setm(w, 'Origin', [0 0 0]);
    setm(w, 'maplatlimit', [-70,-40]);
    setm(w, 'maplonlimit', [10,50]);
    setm(w, 'meridianlabel', 'on')
    setm(w, 'parallellabel', 'on')
    setm(w, 'mlabellocation', 10);
    setm(w, 'plabellocation', 10);
    setm(w, 'mlabelparallel', 0);
    setm(w, 'grid', 'on');
    setm(w, 'frame', 'on');
    setm(w, 'labelrotation', 'on')
    pcolorm(lat,lon,sector_data)
    land = shaperead('landareas', 'UseGeoCoords', true);
    geoshow(w, land, 'FaceColor', [0.5 0.7 0.5])
    colorbar
    caxis(colorlims(variable));
    title(strcat("SWH in the ", sector_name(sector)), 'interpreter','latex','FontSize', 14)
\end{matlabcode}
\begin{center}
\includegraphics[width=\maxwidth{56.196688409433015em}]{figure_2.png}
\end{center}
\begin{matlabcode}
    %colormap turbo
\end{matlabcode}

\vspace{1em}

\matlabheading{7. SWH distance into the ice}


\vspace{1em}
\matlabheading{8. Ice velocities}


\vspace{1em}

\vspace{1em}
\matlabheading{9. Wind and ice velocity relationship}


\vspace{1em}
\matlabheading{10. Ice thickness}

\begin{matlabcode}
variable = "hi";
sector_data = data_format_sector(filedir,variable,sector);
%data = data(sector_mask);
%color_map = seaicecolormap();
%colormap turbo
latitude = [-90,-60];
longitude = [10,50];
figure(2)
w = worldmap('world');
    axesm miller; %, eqaazim eqdazim vperspec, eqdazim flips the x-axis, and y-axis to eqaazim. cassini
    setm(w, 'Origin', [0 0 0]);
    setm(w, 'maplatlimit', [-70,-40]);
    setm(w, 'maplonlimit', [10,50]);
    setm(w, 'meridianlabel', 'on')
    setm(w, 'parallellabel', 'on')
    setm(w, 'mlabellocation', 10);
    setm(w, 'plabellocation', 10);
    setm(w, 'mlabelparallel', 0);
    setm(w, 'grid', 'on');
    setm(w, 'frame', 'on');
    setm(w, 'labelrotation', 'on')
    pcolorm(lat,lon,sector_data)
    land = shaperead('landareas', 'UseGeoCoords', true);
    geoshow(w, land, 'FaceColor', [0.5 0.7 0.5])
    colorbar
    caxis(colorlims(variable));
    title(strcat("Ice thickness in the ", sector_name(sector)), 'interpreter','latex','FontSize', 14)
\end{matlabcode}
\begin{center}
\includegraphics[width=\maxwidth{56.196688409433015em}]{figure_3.png}
\end{center}
\begin{matlabcode}
    %colormap turbo
\end{matlabcode}

\vspace{1em}


\vspace{1em}

\vspace{1em}

\matlabheading{Functions}

\begin{matlabcode}
function coords = sector_coords(sector)
% Coordinates of sector
%   There are three sectors available:
%       SA := South Africa, 
%       WS := Weddell Sea,
%       and EA := East Antarctica
    if sector == "SA"
        coords = [-45,20;-65,20;-45,40;-65,40]; %(NW;NE,SW,SE)
    elseif sector == "EA"
        coords = [];
    elseif sector == "WS"
        coords = [];
    end
end

function sector_data = data_format_sector(filedir,variable,sector,dim)
     if ~exist('dim', 'var')
        dim = 2; 
     end

    % Grid
    grid = "gx1";
    [lat,lon,row] = grid_read(grid);
    % Read in data
    data = data_format(filedir,variable,row,lat,lon,dim);
    ocean_mask = data_format(filedir,'tmask',row,lat,lon);
    % Coordinates
    coords = sector_coords(sector); % (NW;NE;SW;SW) (lat,lon)
    for i = 1:4
        [lat_out(i),lon_out(i)] = lat_lon_finder(coords(i,1),coords(i,2),lat,lon);
    end
    [len,wid] = size(data);
    sector_data = zeros(len,wid);
    sector_mask = false(len,wid);
    sector_data = ~ocean_mask*NaN;
    for i = 0:lat_out(1)-lat_out(2)
        sector_mask(lon_out(1):lon_out(3), lat_out(1)-i:lat_out(3)) = true;
        sector_data(lon_out(1):lon_out(3), lat_out(1)-i:lat_out(3)) = data(lon_out(1):lon_out(3), lat_out(1)-i:lat_out(3));
    end
    
end

function name = sector_name(sector)
    if sector == "SA"
        name = "South African Antarctic";
    elseif sector == "WS"
        name = "Weddell Sea";
    elseif sector == "EA"
        name = "East Antarctic";
    else
        name = "";
        
    end
end

\end{matlabcode}

\end{document}
